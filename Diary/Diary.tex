\documentclass{article}
\usepackage{graphicx} % Required for inserting images

\title{Final Year Project Diary}
\author{Aleksis Vilnitis}
\date{October 2023}

\begin{document}

\maketitle

\section{Project Diary}
\subsection{Week 1}
During week 1 my main objective was to attend a project supervisor meeting and find out as much as I could about the logistics of my project. I wanted to find out what the project entails, and what should I be researching, as well as to ask questions about the project's background. 

Not only did this meeting help me understand the background of my project, but it led me in a general direction of what I should be researching to make a decision on the specifics of my project - what games and puzzles am I going to be solving, using what technologies etc. 

During week 1 I did general research on solving puzzles and other problems using AI technologies, I looked at examples of finished AI puzzle solvers as well as I started working on my project plan. \\



\subsection{Week 2}
Week 2 was for research on constraint satisfaction, rule-based AI algorithms and other AI technologies that could be relevant to the project. The goal of this week was to finish writing the project plan.

Having done general research on the puzzles and games AI could solve, I had to make a decision on what game or puzzle I would be solving for this project. After reading materials on different puzzles, I chose two: Sudoku and Nonogram, both have concrete rules and constraints and both require logic, backtracking and forward thinking to solve them, making them perfect for an AI puzzle solver.

The rest of the week was spent writing the project plan, not only the theoretical abstract but also the project timeline, evaluating risks and referencing.

\subsection{Week 3}
Week 3 Was for doing research and completing the first report of the project - A report on Backtracking and recursion. Having done initial research on deliverables mentioned in the project background, I had a fundamental idea of the concepts, which helped me look for materials.

The first half of the week was spent going through university library materials as well as searching Google Scholar for books, papers and research materials on Backtracking. For recursion, I found an amazing book that not only explained the concept itself but gave examples and an in-depth description of everything to understand recursion.

The second half of the week was spent actually writing the report. I decided to split the two concepts and explain them separately, starting with recursion as that has better materials available. Once finished with recursion, I introduced all the concepts necessary to explain backtracking and was able to complete the report and complete the used references.

\subsection{Week 4}
Week 4 was making myself familiar with test driver development in Python as well as GUI development in Python. This included quite a bit of time spent on tutorials: python libraries tkinter and unittest. 

The first half of the week, as I mentioned I spent following different tutorials on youtube both on how to develop GUIs in Python and how to use TDD principles in Python using the unittest library. Once I had a basic understanding of how both of the libraries work, I was able to start work of my own.

The second half of the week was spent actually developing a GUI. I decided to closely follow the techniques taught in CS2800 and did the TDD cycle until I had a working GUI with a simple label and button. At the end of the week, to make myself more comfortable with the whole project I developed a React web app that lets a user play Sudoku.

\subsection{Week 5}
Most of this week was spent researching ways of actually developing a puzzle solver. The project plan for this week mentions me doing research on what data structures would be needed for an AI puzzle solver. Instead of specifically researching data structures I spent the week watching tutorials on backtracking implementation fundamentals, recursive algorithm design and other relevant algorithm design choices for an AI puzzle solver. 

This week I also attempted to create my first backtracking algorithm - solving the subset sum problem. This was a huge help to my understanding of backtracking. I used not only available resources online, in the library and on Moodle, but I also used available algorithm visualisers to help me understand how the algorithm makes a choice to backtrack to previous states.

After implementing the subset sum solution in my own way I found materials given to us in our Algorithms and Complexity course. It included a chapter of a book with loads of information on backtracking. Using these resources I was able to construct a more formal and tidy solution to the subset sum problem.

\subsection{Week 6}
During week 6 the original plan was to work on relevant data structures. During this week I also had my second supervisor meeting. During this me and my supervisor agreed that I should spend more time on the programming and development side of the project. This week I developed an n-queens solver using backtracking which I then used to start creating a small sudoku solver.

I spent more time working on a GUI for the final project and by the end of the week I had a GUI that loads sudoku games from a text file and lets me play sudoku (without constraints)

\subsection{Week 7}
In my project plan, this week was meant to be for the development of the n-queens problem solver. As I had finished that the week before, I decided to work on the small sudoku solver and my sudoku gui. By the end of this week, I had implemented a constraint-based rule system for the Sudoku game so I couldn't place numbers in cells where it wasn't allowed as well as finished my small sudoku solver that solved an instance of a 3x3 sudoku grid.

\subsection{Week 8}
During week 8 I started working on my report on human solver techniques for Sudoku and Nonogram. I also proposed an updated project plan for the final few weeks of the term until the submission of the interim report. This plan included scaling my small Sudoku solver and implementing it into the GUI so a player can both play and solve the sudoku grid they're playing.

During week 8 I also finished sections about Sudoku solving in the report.

\subsection{Week 9}
This week was spent on report writing. As I still have multiple reports to finish, this week I finished the solving technique report and started working on a report about constraint satisfaction problems. Writing the report on Constraint Satisfaction was difficult as all the materials available were too complex to understand, but once I had gained a basic understanding of the topic from reading easier-to-understand materials online I was able to go through the more difficult materials and complete the report on constraint satisfaction.

This week I also started working on combining all the previously written reports into my interim report submission. This took more time than expected as I had written all reports as individual pieces of research. This meant that each had an abstract, introduction and subsections explaining general concepts for understanding. To combine this into one report I had to make some changes to the reports I was adding to the final document and then format everything based on the provided template.

\newpage

\subsection{Week 10}
During the weekend of Week 9, I started working on the final report for the interim submission "Complexity and NP-Hardness of your puzzle". This was by far the most difficult report yet as this was the topic I was the least familiar with as we had only just covered it in lectures. 

I wanted to make sure that as much as possible is done for my final project supervisor meeting on Monday of week 10, so I completed as much as I could of the Complexity report and added it to the final submission document to show my supervisor. 

After the last supervisor meeting, the rest of Week 10 was spent solely on formatting the interim submission document. Once I had put everything in the document, I revised the original project plan abstract to work for the final submission, wrote an introduction and formatted the sections in an easy-to-understand way that explains the progress made in term 1 step by step.

This week I also had to create the presentation for the demo of what I have done until the interim submission. After proposing a plan for the presentation to my project supervisor and talking over the main points, I created a rough outline that I completed throughout the week. Before the submission of the presentation, I wrote a script and added it to the footnotes to hopefully use them in the presentation.

\subsection{Week 11}
Week 11 was spent finalising the last remaining bits. Complete correct referencing formats in the report, practice the presentation and refactor the code to remove code smells, improve readability and overall ensure everything is ready for submission.

Week 11 is also the week of presenting my work and going over the project report one last time to fix any missed issues.
\end{document}
